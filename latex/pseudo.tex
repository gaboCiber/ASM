\documentclass{report}
\usepackage[spanish]{babel}
\usepackage[left=2.5cm, right=2.5cm, top=3cm, bottom=3cm]{geometry}
\usepackage{amsmath, amsthm, amssymb}
\usepackage[spanish]{babel}
\usepackage{graphicx}
\usepackage{float}
\usepackage{enumerate}
\usepackage{xcolor}
\usepackage{tcolorbox}
\usepackage{listings}

\begin{document}

\section*{Problema 20}

Dada una lista $L$ de $n$ elementos $a_1,a_2,...,a_n$ y un íntervalo $[b,c]$ dentro de esa lista devolver el máximo elemento de ese intervalo.

\begin{itemize}
    \item $b$ : un número de tamaño $db$ que representa $b$
    \item $c$ : un número de tamaño $db$ que representa $c$
    \item $n$ : un número de tamaño $dd$ que representa al tamaño de la lista $L$
    \item $array$ : un array de números de tamaño $dd$ que representa $L$
\end{itemize}


\begin{tcolorbox}
    \lstinputlisting{../codigos/problem20.txt}
\end{tcolorbox}

\section*{Problema 45}

Sea una lista no ordenada $L$ de $n$ elementos $a_1,a_2,...,a_n$ y un número $x$, devolver la suma de todos los números en $L$ que sean mayores que $x$.
\begin{itemize}
    \item $x$ : un número de tamaño $db$ que representa $x$
    \item $n$ : un número de tamaño $dd$ que representa al tamaño de la lista $L$
    \item $array$ : un array de números de tamaño $dd$ que representa $L$
\end{itemize}

\begin{tcolorbox}
    \lstinputlisting{../codigos/problem45.txt}
\end{tcolorbox}


\section*{Problema 66}

Dado dos arrays ordenados $A_1,A_2$ devolver el arreglo ordenado $A_3$ obtenido haciendo una mezcla ordenada de $A_1$ y $A_2$.

\begin{itemize}
    \item $n_1$ : un número de tamaño $dd$ que representa al tamaño de la lista $L$
    \item $array_1$ : un array de números de tamaño $dd$ que representa $L$
    \item $n_2$ : un número de tamaño $dd$ que representa al tamaño de la lista $L^{'}$
    \item $array_2$ : un array de números de tamaño $dd$ que representa $L^{'}$
\end{itemize}

\begin{tcolorbox}
    \lstinputlisting{../codigos/problem66.txt}
\end{tcolorbox}

\newpage

\section*{Problema 81}

Se tiene una lista $L$ de tamaño $n$ con números enteros. Construya la lista $L^\prime$ tal que
$$L^\prime[i]= \max_{j=0}^i L[j]$$
con $i=0,1,...,n-1$.

\begin{itemize}
    \item $n$ : un número de tamaño $dw$ que representa $n$
    \item $L$ : un array de números de tamaño $dw$ que representa $L$
\end{itemize}

\begin{tcolorbox}
    \lstinputlisting{../codigos/problem81.txt}
\end{tcolorbox}

\end{document}